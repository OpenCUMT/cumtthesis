% [本科生]评阅表格,包括:
%  - 本科毕业设计(论文)任务书
%  - 毕业设计(论文)指导教师评阅书
%  - 毕业设计(论文)评阅教师评阅书
%  - 毕业设计(论文)答辩及综合成绩


% 一些可能用到的宏和命令
% \cumt@title@cn - 论文题目
% \cumt@author - 作者姓名
% \cumt@affiliation - 单位(学院)
% \cumt@major - 专业
% \cumt@student@id - 学号
% \cumt@clearpage - 根据输出模式清页

\makeatletter
%============================================================================ 表1
% 中国矿业大学本科毕业设计(论文)任务书
\chapter*{中国矿业大学本科毕业设计(论文)任务书}
\begin{center}
{%
    \renewcommand\arraystretch{1.667}
    \begin{tabularx}{\textwidth}{|>{\centering\arraybackslash}m{2.0cm}|X|>{\centering\arraybackslash}m{2.0cm}|X|}
        \hline
        \multicolumn{4}{|l|}{设计(论文)题目: \cumt@title@cn} \\
        \hline
        学院 & \cumt@affiliation & 专业年级 & \cumt@major \\
        \hline
        学生姓名 & \cumt@author & 学号 & \cumt@student@id \\
        \hline
        \multicolumn{4}{|l|}{%
            \parbox[t][9cm][l]{\textwidth-2\ccwd}{%
                1、设计(论文)的主要内容

                %******************************
                % 在此处填写<设计(论文)的主要内容>
                %******************************

            }
        } \\
        \hline
        \multicolumn{4}{|l|}{%
            \parbox[t][9cm][l]{\textwidth-2\ccwd}{%
                2、设计(论文)的基本要求

                %******************************
                % 在此处填写<设计(论文)的基本要求>
                %******************************

                \vfill
                \hspace{15\ccwd}指导教师签字:% \includegraphics[width=2.0cm]{path/to/signature.png}
                \vspace{12bp}
            }
        } \\
        \hline
    \end{tabularx}

}
\end{center}
\clearpage
%============================================================================
%============================================================================ 表2
% 中国矿业大学毕业设计(论文)指导教师评阅书
\chapter*{中国矿业大学毕业设计(论文)指导教师评阅书}
\begin{center}
{%
    \renewcommand\arraystretch{1.667}
    \begin{tabularx}{\textwidth}{|>{\centering\arraybackslash}m{3.5cm}|X|>{\centering\arraybackslash}m{2.0cm}|X|}
        \hline
        学生姓名 & \cumt@author & 学号 & \cumt@student@id \\
        \hline
        设计(论文)题目 & \multicolumn{3}{c|}{\cumt@title@cn} \\
        \hline
        \multicolumn{4}{|l|}{%
            \parbox[t][20cm][l]{\textwidth-2\ccwd}{%
                指导教师评语(①基础理论及基本技能的掌握;②独立解决实际问题的能力;③研究内容的理论依据和技术方法;④取得的主要成果及创新点;⑤工作态度及工作量;⑥总体评价及建议成绩;⑦存在问题;⑧是否同意答辩等):

                %*************************
                % 在此处填写<指导教师评语>
                %*************************

                \vfill
                成绩:\hspace{12\ccwd}指导教师签字:% \includegraphics[width=2.0cm]{path/to/signature.png}

                \hspace{20\ccwd} \hspace{2\ccwd}年\hspace{\ccwd}月\hspace{\ccwd}日
                \vspace{12bp}
            }
        } \\
        \hline
    \end{tabularx}

}
\end{center}
\clearpage
%============================================================================
%============================================================================ 表3
% 中国矿业大学毕业设计(论文)评阅教师评阅书
\chapter*{中国矿业大学毕业设计(论文)评阅教师评阅书}
\begin{center}
{%
    \renewcommand\arraystretch{1.667}
    \begin{tabularx}{\textwidth}{|>{\centering\arraybackslash}m{3.5cm}|X|>{\centering\arraybackslash}m{2.0cm}|X|}
        \hline
        学生姓名 & \cumt@author & 学号 & \cumt@student@id \\
        \hline
        设计(论文)题目 & \multicolumn{3}{c|}{\cumt@title@cn} \\
        \hline
        \multicolumn{4}{|l|}{%
            \parbox[t][20cm][l]{\textwidth-2\ccwd}{%
                评阅教师评语(①选题的意义;②基础理论及基本技能的掌握;③综合运用所学知识解决实际问题的能力;④工作量的大小;⑤取得的主要成果及创新点;⑥写作的规范程度;⑦总体评价及建议成绩;⑧存在问题;⑨是否同意答辩等):

                %*************************
                % 在此处填写<评阅教师评语>
                %*************************

                \vfill
                成绩:\hspace{12\ccwd}评阅教师签字:% \includegraphics[width=2.0cm]{path/to/signature.png}

                \hspace{20\ccwd} \hspace{2\ccwd}年\hspace{\ccwd}月\hspace{\ccwd}日
                \vspace{12bp}
            }
        } \\
        \hline
    \end{tabularx}

}
\end{center}
\clearpage
%============================================================================
%============================================================================ 表4
% 中国矿业大学毕业设计(论文)答辩及综合成绩
\chapter*{中国矿业大学毕业设计(论文)答辩及综合成绩}
\begin{center}
{%
    \renewcommand\arraystretch{1.3}
    \begin{tabularx}{\textwidth}{|>{\centering\arraybackslash}m{7cm}|X|X|X|X|X|}
        \hline
        
        \multicolumn{6}{|c|}{答\hspace{2\ccwd}辩\hspace{2\ccwd}情\hspace{2\ccwd}况} \\
        \hline
        \multirow{2}{*}[-3.5ex]{提\hspace{\ccwd}出\hspace{\ccwd}问\hspace{\ccwd}题} & \multicolumn{5}{c|}{回\hspace{\ccwd}答\hspace{\ccwd}问\hspace{\ccwd}题} \\
        \cline{2-6}
        & 正确 & 基本正确 & 有一般性错误 & 有原则性错误 & 没有回答 \\
        \hline
        {} & {$\checkmark$} & {} & {} & {} & {} \\
        \hline
        {} & {$\checkmark$} & {} & {} & {} & {} \\
        \hline
        {} & {$\checkmark$} & {} & {} & {} & {} \\
        \hline
        {} & {$\checkmark$} & {} & {} & {} & {} \\
        \hline
        \multicolumn{6}{|c|}{
            \parbox[t][7cm][l]{\textwidth-2\ccwd}{%
                答辩委员会评语及建议成绩:

                %******************************
                % 在此处填写<答辩委员会评语>
                %******************************
                \vfill
                成绩:\hspace{12\ccwd}答辩委员会主任签字:% \includegraphics[width=2.0cm]{path/to/signature.png}

                \hspace{20\ccwd} \hspace{2\ccwd}年\hspace{\ccwd}月\hspace{\ccwd}日
                \vspace{12bp}
            }
        } \\
        \hline
        \multicolumn{6}{|l|}{成绩评定: ** }\\
        \hline
        \end{tabularx}
        \vskip -1.4pt
        \begin{tabularx}{\textwidth}{|X|X|X|X|X|X|}
        \hline
        \multicolumn{6}{|c|}{成绩评定} \\
        \hline
        成绩组成 & 指导教师 & 评阅教师 & 答辩成绩 & 其他 & 总评 \\
        \hline
        成绩比例 & {} & {} & {} & {} &  \multirow{2}{*}{}\\
        \cline{1-5}
        评分 & {} & {} & {} & {} &  \\
        \hline
        \multicolumn{6}{|c|}{
            \parbox[c][2cm][l]{\textwidth-2\ccwd}{%
                \vfill
                \hspace{15\ccwd}学院领导签字:% \includegraphics[width=2.0cm]{path/to/signature.png}
                \vfill
                \hspace{17\ccwd} \hspace{2\ccwd}年\hspace{\ccwd}月\hspace{\ccwd}日
                \vfill
            }%
        } \\
        \hline
    \end{tabularx}
}
\end{center}
\cumt@clearpage
%============================================================================
