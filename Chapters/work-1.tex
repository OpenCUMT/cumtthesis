\bichapter{模板使用说明}{Template Usage Guide}
\bisection{简介}{Overview}

该 \LaTeX 模板按照\href{https://gs.cumt.edu.cn/info/1049/3149.htm}{《中国矿业大学研究生学位论文撰写规定及模板(2021年版)》}的要求编写(以下简称《撰写规定》)。目前支持硕士、博士毕业论文撰写(暂不支持外文学院硕士毕业论文)。

\bisection{文档类}{Document Class and Options}
在文档开始时,需指定文档类为:
\begin{center}
    \texttt{\textbackslash documentclass\{cumt-graduate-thesis\}}
\end{center}


\bisection{元数据配置}{Metadata Configuration}
\label{section: metadata configuration}
在文档的导言区,使用 \texttt{\textbackslash cumtsetup\{\}} 配置论文的元数据。以下是一些关键配置项:

\begin{itemize}
    \item \texttt{output}:设置输出格式,可选 \texttt{electronic}(电子版)或 \texttt{print}(打印版)。
    \item \texttt{funding-on-cover}:是否在封面显示基金信息,可选 \texttt{true} 或 \texttt{false}。
    \item \texttt{title} 和 \texttt{title*}:分别设置中文标题和英文标题。
    \item \texttt{author}:设置作者姓名。
    \item \texttt{thesis-name}:设置论文类型,如“硕士学位论文”或“博士学位论文”。
    \item \texttt{supervisor} 和 \texttt{supervisor-title}:设置导师姓名和职称。
    \item \texttt{co-supervisor} 和 \texttt{co-supervisor-title}:设置第二导师姓名和职称(选填)。
    \item \texttt{year} 和 \texttt{month}:设置论文提交的年月。
    \item \texttt{degree-applied}:设置申请学位的名称。
    \item \texttt{affiliation}:设置培养单位。
    \item \texttt{major}:设置学科专业。
    \item \texttt{field}:设置研究方向。
    \item \texttt{defense-committee-chair}:设置答辩委员会主席。
    \item \texttt{reviewer}:设置评阅人。
    \item \texttt{security-level}:设置密级,默认为“公开”。
    \item \texttt{clc}:设置中图分类号。
    \item \texttt{udc}:设置 UDC 分类号(选填)。
    \item \texttt{funding}:设置资助信息,格式为 \texttt{\{<基金名称1>(编号1);<基金名称2>(编号2);...\}}。
    \item \texttt{degree-category}:设置学位类别。
    \item \texttt{degree-level}:设置学位级别。
    \item \texttt{language}:设置论文语种,默认为“中文”。
    \item \texttt{student-id}:设置学号。
    \item \texttt{program-duration}:设置学制。
    \item \texttt{defense-committee-members}:设置答辩委员会成员(选填)。
    \item \texttt{electronic-thesis-format}:设置电子版论文格式(选填)。
    \item \texttt{electronic-thesis-publisher}:设置电子版论文出版者(选填)。
    \item \texttt{electronic-thesis-publisher-location}:设置电子版论文出版地(选填)。
    \item \texttt{permission-statement}:设置授权声明(选填)。
\end{itemize}

\bisection{多级标题}{Titles}
使用以下命令输入各级标题:
\begin{itemize}[itemsep=2pt,topsep=5pt]
    \item \verb|\bichapter{}{}|:中文、英文一级标题(章)。
    \item \verb|\bisection{}{}|:中文、英文二级标题(节)。
    \item \verb|\subsection{}|:三级标题(小节)。
    \item \verb|\subsubsection{}|:四级标题(次小节)\footnote{《撰写规定》中未对四、五、六级标题作具体要求}。
    \item \verb|\paragraph{}|:五级标题(段落)。默认为内嵌(Run-in)标题,可使用\verb|\paragraph*{}|实现陈列(Display)标题。
    \item \verb|\subparagraph{}|:六级标题(小段)。默认为内嵌(Run-in)标题,可使用\verb|\subparagraph*{}|实现陈列(Display)标题。
\end{itemize}

\bisection{内容和章节}{Contents and Chapters}

\subsection{封面与扉页}
\begin{itemize}[itemsep=2pt,topsep=5pt]
    \item \verb|\MakeCoverPage|:生成封面页。
    \item \verb|\MakeTitlePage|:生成扉页。
\end{itemize}

封面与扉页信息来自于元数据(见\ref{section: metadata configuration}),因此请确保所有必要字段都已正确填写。

\subsection{学位论文使用授权声明}
通过\verb|\MakeCopyrightPage|命令生成。

\subsection{致谢}
致谢部分放置在\texttt{acknowledgements.tex},使用\texttt{acknowledgements}环境。

\subsection{摘要与关键词}
\begin{itemize}[itemsep=2pt,topsep=5pt]
    \item \verb|\begin{cnabstract}...\end{cnabstract}|:定义中文摘要。
    \item \verb|\begin{enabstract}...\end{enabstract}|:定义英文摘要。
    \item \verb|\cnkeywords{}|:定义中文关键词(需要被包裹于\texttt{cnabstract}环境内)。
    \item \verb|\enkeywords{}|:定义英文关键词(需要被包裹于\texttt{enabstract}环境内)。
    \item \verb|\MakeCnAbstract|:生成中文摘要页面。
    \item \verb|\MakeEnAbstract|:生成英文摘要页面。
\end{itemize}

\subsection{目录}
\begin{itemize}[itemsep=2pt,topsep=5pt]
    \item \verb|\MakeCnContents|:生成中文目录。
    \item \verb|\MakeEnContents|:生成英文目录。
\end{itemize}

\subsection{图表清单}
\begin{itemize}[itemsep=2pt,topsep=5pt]
    \item \verb|\MakeListOfFigures|:生成图清单。
    \item \verb|\MakeListOfTables|:生成表清单。
\end{itemize}

\subsection{变量注释表}
变量注释表放置在\texttt{denotation.tex},使用\texttt{denotation}环境,并使用\verb|\Variable{}{}|命令依次添加符号和解释。

\subsection{正文内容}
正文部分每个章节应保存为单独的.tex文件,并通过\verb|\input{}|命令引入主文档中。

\subsection{脚注}
使用\verb|\footnote{}|命令插入脚注\footnote{脚注示例}。

\subsection{参考文献}
参考文献列表由\verb|\MakeReferencePage|命令生成,并按照GB/T-7714标准进行引用格式化。使用\verb|\addbibresource{}|在导言区指定bib文件。

\subsection{附录}
附录用于包含补充材料(非必须)。该部分放置在\texttt{appendix.tex},需使用\texttt{appendix}环境。

\subsection{简历}
简历部分放置在\texttt{resume.tex},需使用\texttt{resume}环境\footnote{使用\texttt{\textbackslash resumesection{}}命令添加小节标题。}。

\subsection{原创性声明}
原创性声明页可通过\verb|\MakeDeclarationPage|命令生成。

\subsection{学位论文数据集}
学位论文数据集页面由\verb|\MakeDataCollectionPage|命令生成,其中包含了论文的各项元数据信息,如密级、分类号、资助项目等。

\bisection{浮动体}{Floats}
\subsection{图}
模板使用 \textsf{subcaption} 宏包实现子图。具体用法参考\url{https://mirrors.ibiblio.org/CTAN/macros/latex/contrib/caption/subcaption.pdf}

\subsection{表}
......

\subsection{算法}
模板中使用 \texttt{algorithm2e} 宏包实现算法环境。具体用法参考官方文档。

\begin{algorithm}
  \SetAlgoLined
  \KwData{this text}
  \KwResult{some text }
  $x:=x_{0}$\;
  \While{$x<100$}{
    $x:=y^2$\;
    \eIf{$x>a$}{
      $y:=y-1$\;
      $c:=10290$\;
    }{
      $y:=y/2$\;
    }
  }
  \caption{**算法}
  \label{algo:algorithm1}
\end{algorithm}

\nocite{*}