\bichapter{结论}{Conclusion}
本文从自然因素、外部环境和内部结构等方面,详细分析了影响我国煤炭供给和需求的因素,探索煤炭供需与其影响因素的规律,构建了我国煤炭供需预测预警指标体系,对我国煤炭供需进行预测预警。

我国的煤炭供给受许多因素度影响,而且随着时间的推移,出现新的特点。

……

……

……

……

……

……

……

……

……

……

……

……

……

……

……

目前,我国的铁路运输压力又所缓解,但铁路运输还是制约着我国的煤炭供给。我国煤炭资源区域分异现象与经济区域分异性相悖,由此造成了“西煤东调”和“北煤南运”的运输格局,这种能源中心与经济中心的差异性,形成了大量的煤炭运输需求以及非常集中的煤炭流量,但因资金的缺口及体制的原因,铁路运输现在将来一段时期都制约着我国的煤炭供给。

目前,我国的铁路运输压力又所缓解,但铁路运输还是制约着我国的煤炭供给。我国煤炭资源区域分异现象与经济区域分异性相悖,由此造成了“西煤东调”和“北煤南运”的运输格局,这种能源中心与经济中心的差异性,形成了大量的煤炭运输需求以及非常集中的煤炭流量,但因资金的缺口及体制的原因,铁路运输现在将来一段时期都制约着我国的煤炭供给。

目前,我国的铁路运输压力又所缓解,但铁路运输还是制约着我国的煤炭供给。我国煤炭资源区域分异现象与经济区域分异性相悖,由此造成了“西煤东调”和“北煤南运”的运输格局,这种能源中心与经济中心的差异性,形成了大量的煤炭运输需求以及非常集中的煤炭流量,但因资金的缺口及体制的原因,铁路运输现在将来一段时期都制约着我国的煤炭供给。


……

……

……

……

……

……

……

……

……

……

……

……

……

……

……

……

……

……

……

……

……

……

……

……

……

……

……

……

……

……

……

……

……

……

……

……

……

……

……

……

……

……

……

……

……
